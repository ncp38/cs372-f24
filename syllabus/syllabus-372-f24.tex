\documentclass [letterpaper,11pt]{article}
\usepackage{fullpage,amsmath,hyperref}
\hypersetup{colorlinks,urlcolor=blue}

\newcommand{\urlnofont}[1]{\urlstyle{same}\url{#1}}

\begin{document}

\begin{center}
\large COMP 372 --- Artificial Intelligence --- Fall 2024
\end{center}

\noindent\begin{tabular}{@{}ll}
\textbf{Instructor:} & Nate Phillips \\
\textbf{Times:} & 9:30-10:45 AM Tuesday/Thursday\\
\textbf{Classroom:} & Briggs 119 \\
\textbf{Course website:} & \url{http://www.cs.rhodes.edu/AI}\\
\textbf{Email:} &phillipsn@rhodes.edu\\
\textbf{Office:} & Briggs 210\\
\textbf{Office hours:} & Tues/Thurs 3:15-4:15 PM \& Wed 1-3 PM \\ &Also available by appointment and via Zoom\\
\end{tabular}
\begin{description}

\item[Course Overview:]
This course presents an overview of the fundamental techniques used in artificial intelligence today. Topics will include a subset of the following: agents, intelligent search, constraint satisfaction, game playing, utility theory, decision making under uncertainty, reinforcement learning, probabilistic reasoning, machine learning, and neural networks. Other topics may be presented with student interest and as time permits.

\item[Text:]
Russell and Norvig, \emph{Artificial Intelligence: A Modern Approach}, 4th edition, Pearson, 2021. (You may also use the older 3rd edition.)

\item[Prerequisites:]
COMP 241 (Data Structures and Algorithms) is required. COMP 172 (Discrete Structures) is highly suggested, but not required. In particular, knowledge of the data structures taught in 241 (especially sets, maps, priority queues, and graphs) will be assumed, as will the mathematical concepts (such as elementary probability theory) taught in 172.



\item[Tentative Course Schedule:]\
\begin{itemize} \setlength{\itemsep}{0em}\setlength{\parskip}{0pt}
	\item Uninformed search (DFS, BFS, uniform-cost search)
	\item Heuristic search (A* algorithm)
	\item Adversarial search (Minimax algorithm, alpha-beta pruning)
	\item Probabilistic reasoning
	\item Bayesian networks
	\item Naive Bayes classifiers
	\item Markov chains and hidden Markov models
	\item Reinforcement learning (value iteration, Q-learning)
	\item Neural networks
	\end{itemize}





\item[Coursework:] \

\begin{tabular}{lcc} 
& Tentative weight & Tentative date \\ \hline
Homework & 15\% & \\
Programming projects & 35\% & \\
Papers & 15\% & \\
Midterm & 15\% & Thursday, October 29, in class\\
Comprehensive final exam & 20\% &  Friday, December 13, 8:30 AM\\
\end{tabular}

Grades of A--, B--, C--, and D-- are correlated with final course grades of 90\%, 80\%,
70\%, and 60\%, respectively.  If your final course grade falls near a letter grade boundary,
I may take into account participation, attendance, and/or improvement during the semester.

Assignments are due at 11:59pm unless otherwise specified.  \textbf{Late work is unprofessional} and will generally be \textbf{penalized at a rate of 8\% per 24 hours late} or 1 point per day for homework assignments.  If the assignment is submitted three days after it is due, it will not be accepted except in special cases, which might include sickness, injury, or other unforeseen events. In these situations, you should contact the professor as soon as is feasible to come up with a plan for submission.

\item[Office Hours:]
In addition to regular office hours, I am often available immediately before or after class for 
short questions.  You never need an appointment to see me during regular office hours; you
can just come by.  

\textbf{Don't be shy about coming by my office or sending me emails!!}  My goal for this course is to help you understand course concepts and reach your full potential. I think proactive communication is important and necessary in achieving this goal.


\item[Attendance:]
Prompt attendance is expected for each class and is necessary for your success. If you know ahead of time that you will be late or
absent from class for any reason, please discuss this with me at the beginning of the semester
or as early as possible. Otherwise, if your attendance deteriorates, you may be referred to the dean and asked to drop the course. Note that in-class assignments or bonuses, if missed, will not be available to retake.

Attendance and participation may be considered when assigning a final grade.
Attendance will be checked each class lecture period, with consequences for \textbf{excessive absences (more than five)}.  For each additional absence, points will be deducted from your grade, up to and including \textbf{dropping a letter grade}.


\item[Workload:]
It is important to stay current with the material.  You should be prepared to devote at least 2--3 hours outside of class for each in-class lecture.  In particular, you should expect to spend a significant amount of time for this course working on a computer trying example programs and developing programming assignments. For the larger/more involved programming assignments (the projects), it is expected to take \textbf{an average of 20 hours} to complete. Do not wait until the last minute to start your programming assignments. 

Even with intentional focus and preparation, it is possible for issues to come up that delay assignment submission.  As such, a \textbf{once-a-semester extension} is available upon request, allowing for up to 48 hours' grace for submissions. This extension may not be used for the last project in the semester but is otherwise available.

You are encouraged to form study groups with colleagues from the class. The goal of these groups is to clarify and solidify your understanding of the concepts presented in class, and to provide for a richer and more engaging learning experience. However, you are expected to turn in your own code that represents the results of your own effort. 

\item[Getting Help:]
Students can get help with their coursework several different ways. The first place to
reach out for help is to come to office hours. Scheduled office hours are listed on Canvas, the class website, and this syllabus.
Additional meeting times are available by request. You may also email the instructor.  Tutoring is also available, though tutors may not have specific expertise in AI topics.
\\\\
\item[Programming Assignments:]\

\begin{itemize}\setlength{\itemsep}{0em}\setlength{\parskip}{0pt}
	\item All programs assigned in this course must be written in Java, unless otherwise specified.  When turning in assignments,
	submit only the Java source code files (\texttt{.java}) and other files necessary to allow your program to run (image files, data sets, etc.). Do not submit any files generated by the IDE (e.g., \texttt{.class}).	
			\item I recommend backing up your code somewhere as you're working on your assignments.  Computer
		crashes or internet downtime are not valid excuses for missing a deadline.
				
\item Grades for programming projects will be based on correctness of the program output, efficiency and appropriateness of the algorithms used in the code, and style/documentation of the source code.  In addition, bonus points may be offered on assignments for meeting particular challenges or for creative or excellent program functionality.

\item To ensure your programs recieve as high a score as possible:
	\begin{itemize}
	\item  \textbf{Start work early}, and turn your programs in on time!  Even if you are struggling to resolve a minor issue, it is often better to turn the program in on time than to spend a few extra days working on the issue.
	\item 	\textbf{Look at the programming assignment page}--typically, there is a clear list of milestones your program needs to accomplish and documents to turn in.  This is a good list to review before you start and before you submit your assignment.
	\item  \textbf{Create a plan.} How are you going to:
\begin{itemize}
\item Budget time for your program in light of your other commitments?
\item Use what you've learned to solve the problem?  Often, thinking out and designing the program beforehand results in a much better (and faster!) process.
\end{itemize}
	\item	\textbf{Document any errors} you are aware of in your program!  This enables your instructor to help you understand/address the underlying issue and is a sign of having tested your program well.
\item 	Finally, don't hesitate to \textbf{come see me or talk to the tutors} if you get stuck. We're here to help you understand and grow in your programming knowledge!
\end{itemize}
\end{itemize}

\item[Rules for Completing Assignments Independently]\
\begin{itemize}
        \item Unless otherwise specified, programming assignments handed in for this course are to be done \emph{independently}.  
        \item Talking to people (faculty, other students in the course, others with programming experience) is one of the best ways to learn.  I am willing to answer your questions or provide hints if you are stuck, and the tutors are an excellent resource for programming help and instruction as well.  But when you ask other people for help, sometimes
        it is difficult to know what constitutes legitimate assistance and what does not.  In general, follow these rules:
        
        \begin{itemize}
                \item \textbf{Rule 1: Do not look at anyone else's code for the same project, or a different project that solves a similar 
                or identical problem.}
                
                Details: ``Anyone else'' here refers to other members of the class, people who have taken the class before, people at other
                schools enrolled in similar classes, or any code you find online or in print.  ``Similar or identical problem'' here should 
                allow you to look at code that uses techniques applied in different situations that you can then 
                adapt to your project.  However, if you find yourself copying-and-pasting code or directly transforming
                code line by line to fit into your program, then that is considered plagiarism.  This includes code from the internet, automated programs like ChatGPT, or other people; as such, be careful that your programs are your own work and aren't copied from other sources.\newline
                              
                Exception: You may help someone else debug their program, or seek assistance in debugging yours.  However, 
                this requires the person writing the code being debugged to have made a good-faith attempt
                to write the program in the first place, and the goal of the debugging must be to fix
                one specific problem with the code, not re-write something from scratch.\newline
                
                \item \textbf{Rule 2: Do not write code or pseudocode with anyone else.}
                
                Details: You must make a good faith effort to develop and implement your ideas
                independently before seeking assistance.  Feel free to discuss the project \emph{in general} with anyone else
                before you begin and as you're developing your program, but when you get to the level of writing code or
                pseudocode, you should be working independently.
                
                        \end{itemize}
        \end{itemize}

\textbf{A violation of these rules constitutes plagiarism and is not acceptable behaviour for Rhodes
students.} Such violations will be dealt with harshly and will have consequences that are
reflected in your grades, which can include referral to the honor council, failing the assignment, dropping a letter grade for
the entire course, or even failing the course.  However, if you have any questions about what
is acceptable for this course, please send me an email with the details relating to your specific
case.

At any point, you may be asked to explain your code or ideas or to reflect on aspects of your
coding style, approach, or the assignment itself. This self-analysis is part of your assignment,
but, even more, thoughtful responses will help you better understand your own code and
design practices!

The underlying idea is that course work and outside assistance should genuinely help you to
learn the material (as opposed to just getting the assignment done). Programming assignments
are graded as a benefit to you; they are your chance to show what you have learned
under circumstances less stressful than an exam. In return, I ask only that your work fairly
reflect your understanding and your effort in the course.

\item[Coding Style:]
Designing algorithms and writing the corresponding code is not a dry, mechanical process, but an art form.  Well-written code has an aesthetic appeal while poor form can make other programmers (and instructors) cringe. Programming assignments will be graded based on correctness and style. To receive full credit for graded programs, you must adhere to good programming practices. Therefore, \textbf{your assignment must contain the following:}
\begin{itemize}\setlength{\itemsep}{0em}\setlength{\parskip}{0pt}
	\item A comment at the top of the program that includes the author of the program,
	the date, and a brief description of what the program does.
	\item Concise comments that summarize major sections of your code, along with a comment
	for each function in your code that describes what the function does.
	\item Meaningful variable and function names
	\item Well-organized code
	\item White space or comments to improve legibility
\end{itemize}

\item[Class Conduct:] \
   \begin{itemize}\setlength{\itemsep}{0em}\setlength{\parskip}{0pt}
   	\item I encourage everyone to participate in class.  Raise your hand if you have
	a question or comment.  Please don't be shy about this; if you are confused about
	something, it is likely that someone else is confused as well.
		Teaching and learning is a partnership between the instructor and the students, and asking questions not only helps you understand the material, it also helps me better connect with you in my teaching.
	\item When in class, only use cell phones or other electronic devices for classwork and keep the volume on silent.
	\item  If you cannot make it to class for whatever reason, make sure that you know what happened during the lecture that you missed. The information from that class is your responsibility. (A good way of doing this is to ask a classmate!)
     \item  If you have to leave a class early or are going to be late, inform the instructor in advance. 
     \end{itemize}
     
\item[Additional policies:] \
On the class webpage, there are additional class and college policies covering accommodations, academic integrity, diversity, sexual misconduct disclosure, and recording lectures.  
   
   
\end{description}

   \end{document}
