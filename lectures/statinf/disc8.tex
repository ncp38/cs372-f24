\documentclass[letterpaper,11pt,english,oneside]{article}
\usepackage[pdftex]{graphics}
\usepackage{amsmath,amssymb,rotating,multirow}
\usepackage{palatino}
\usepackage{url,fullpage}

\usepackage[plain]{algorithm}
\usepackage{algpseudocode}

\pagestyle{empty}

\usepackage{parskip}
\usepackage{setspace}
\onehalfspacing

\setlength{\parskip}{10pt}
\setlength{\parindent}{0em}

\newcommand{\prob}[1]{\mathbb P \left [ #1 \right ]}
\newcommand{\g}{\,|\,}

\begin{document}

Names:

\begin{center}
\LARGE CMPSCI 240\\
Reasoning Under Uncertainty\\
Discussion 8\\
\end{center}

The writers for \textit{Saturday Night Live} are analyzing some
classic \textit{SNL} sketches from the 1970s to figure out which
combinations of actors worked best together to create the funniest
sketches.

In a sample of 10 classic sketches, viewers thought 6 were funny and 4
were not funny. In the 6 funny sketches, Dan Aykroyd appeared in 4,
John Belushi in 2, and Jane Curtin in 5.  In the 4 non-funny sketches,
Dan Aykroyd appeared in 3, John Belushi in 1, and Jane Curtin in 2.

The writers assume that the presence or absence of any actor is
conditionally independent of the presence or absence of any other
actor, given that the sketch is already known to be funny or not.

The writers have just created a new sketch involving Dan Aykroyd and
Jane Curtin but not John Belushi.  \textbf{What is the maximum a
  posteriori hypothesis regarding whether this sketch will be funny or
  not.  Then find the posterior probability that this sketch will be
  funny.}

Note: You should smooth the probabilities of the features given the
classes, but not the priors.

\end{document}
