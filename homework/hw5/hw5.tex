\documentclass[letterpaper,11pt,english,oneside]{article}
\usepackage[pdftex]{graphics}
\usepackage{amsmath,amssymb}
\usepackage{url,fullpage}

\usepackage[plain]{algorithm}
\usepackage{algpseudocode}

\pagestyle{empty}

\usepackage{parskip}
\usepackage{setspace}
%\onehalfspacing

%\setlength{\parskip}{10pt}
%\setlength{\parindent}{0em}

\newcommand{\prob}[1]{\mathbb P \left [ #1 \right ]}
\newcommand{\g}{\,|\,}

\begin{document}

Artificial Intelligence \\
Homework 4


\begin{enumerate}


\item \emph{Note: This problem involves multiplying vectors and matrices.  You are welcome to use software to do the
multiplications itself, however, you should show all your work.  }

%Another way of thinking about this is that you should 
%understand the math well enough to do it by hand if I asked you to.  That is, you should know how to multiply 2-by-2
%matrices and vectors by hand.  (Hint, hint.)}


Professor Somnus is investigating whether the students in his class are getting enough 
sleep.  He collects some data and deduces that the probability that a student will get
enough sleep on a given night only depends on whether they got enough sleep the night before.
If a student gets enough sleep the previous night, the probability they will get enough sleep
tonight is 0.8.  If they didn't get enough sleep the previous night, the probability they
will get enough sleep tonight is only 0.3.  

\begin{enumerate}

	\item Formulate this problem using a Markov chain. 
	
	(1) Draw the Markov chain diagram 
	showing the probability of transitioning between ``enough sleep'' and ``not 
	enough sleep.'' (This is the not the Bayes net diagram; this is the diagram that shows up
	at the top of the page at en.wikipedia.org/wiki/Markov$\_$chain.)
	
	(2) Write down the transition matrix $T$.
	
	\item Assuming you got enough sleep on night 0, what's the probability you get enough
	sleep on night 3?
	
	(1) Show the initial state vector, $v_0$.
	
	(2) Calculate and show the probability distribution for nights 1, 2, and 3 by multiplying by $T$.
	
	\item In the far, far, future, what is the probability of getting enough sleep on 
	some night?

\end{enumerate}

\item Continuing with the situation in Problem 1, Professor Somnus has no way of directly
observing whether or not his students are getting enough sleep.  All he can observe is
whether they are falling asleep in his class or not.  He knows that if a student gets
enough sleep on some night, the next day there is a 0.1 probability that they will fall 
asleep in class.  If they don't get enough sleep the night before, there's a 0.3
probability of falling asleep in class.

For this question, assume that night $x$ is followed by day $x$. 

\begin{enumerate}

\item Professor Somnus observes a certain student falling asleep in class on day 1, staying
awake on day 2, but falling asleep again on day 3.  Calculate the probability, using the forward algorithm, that the student got enough sleep on night 3, given that sequence of observations.  (Remember, night 3 happens right before day 3).  The professor assumes there's
an equal prior probability of enough sleep/not enough sleep on night zero.

(1) Write out the appropriate matrices $O_1$, $O_2$, and $O_3$.

(2) Write out the initial vector $f_0$.

(3) Use the forward algorithm to calculate the probability distributions for enough sleep/not
enough sleep on nights 1, 2, and 3.  Show your work.

(Note that there is another page!)

\item Suppose Prof Somnus wants to predict the probability distribution for enough
sleep/not enough sleep on night \textbf{4}, given the evidence on days 1, 2, and 3.  (Note that there is no evidence observation for day 4.)
How do you think the professor could do this?  Hint: If there is no evidence, an HMM simply
degrades back into a Markov chain.  What should Prof Somnus predict as the probability
distribution for for enough sleep/not
enough sleep  on day 4?  Show your work.

\item Professor Somnus remembers that in Hidden Markov Models, the observations can affect
our beliefs about the past, as well as the future.  Use the backward algorithm to re-calculate
Professor Somnus's belief of getting enough sleep on night 1, given the the
sequence of observations.  Show your work.  (Hint: Your work for part (a) matters here, but part (b) does not.)


\end{enumerate}


\end{enumerate}

\end{document}
